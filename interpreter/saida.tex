\documentclass[12pt]{article} 
\usepackage[latin1,utf8]{inputenc} 
\usepackage[portuges]{babel} 
\usepackage[pdftex]{graphicx}  
\graphicspath{{figures/}}  

\begin{document} 
\begin{abstract} 
Resumo simples aqui
\end{abstract}

Primeira linha 
[ linkteste.com.br ] {linkteste.com.br}

\begin{equation} 
1 + 2
\end{equation}

Blabla dfjasldkj sdfks
teste novo

Blabla
wrongword  aqui temos um warning com uma stopword
``aspas"
esquerda ``aspas" direita
``aspas" direita
esquerda ``aspas"
``teste1", ``teste2", ``teste3", ``teste4", ``teste5"\textit{ teste}

\textit{ italico}
\textit{`` italico com aspas "}
\textit{palavra a esquerda ``aspas"}
\textit{``aspas" palavra a direita}
\textit{esquerda ``meio aspas" direita}
\textbf{ negrito }
\textbf{``negrito com aspas"}
\textbf{ esquerda ``aspas"}
\textbf{``aspas" direita}
\textbf{ esquerda ``aspas" direita}
Deixar possivel utilizar as duas formas
\textit{ \textbf{  italico e negrito }}
\textbf{ \textit{ negrito e italico} }
\textbf{\textit{ terceira forma de escrever italico e negrito }}

\textbf{\textit{``negrito aspas"}}
\textbf{\textit{``esquerda aspas" expressao }}
\textbf{\textit{ expressao ``direita aspas"}}
\textbf{\textit{ esquerda ``aspas" direita }}
a+b
\begin{enumerate} 
\item teste 1
\item teste 2
\end{enumerate}

22 plus 3 equals:
25

22 minus 3 equals:
19

22 div 3 equals:
7.33333333333

22 times 3 equals: 
66

22 plus 10 times 3 equals: 
52
\end{document}

